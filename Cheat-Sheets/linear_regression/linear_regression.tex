\documentclass[a4paper,12pt,ngerman,fleqn]{article}

    \renewcommand{\familydefault}{\sfdefault}
    \usepackage[a4paper, margin={0cm,0cm},twocolumn, layouthoffset=0pt]{geometry}
    
    \usepackage{tikz}
    \usepackage{framed}
    \usepackage{amsmath}
    \setlength{\mathindent}{0pt}
    \usepackage{amsfonts}
    \usepackage{amssymb}
    \usepackage{amsmath}
    \usepackage{tabularx, colortbl}

    \usepackage{xcolor}
    \definecolor{accent}{HTML}{0eb7ac}

    \linespread{1}
    \vspace{1cm}
    \renewcommand{\arraystretch}{2}
    \arrayrulecolor{accent}
    \setlength\arrayrulewidth{1.5pt}
    \newcommand{\mybox}[3]{
        \centering
        \begin{tabularx}{0.9\textwidth}{|X|}
            \rowcolor{accent}
            \textcolor{white}{\textbf{#1}} \\
            \def\temp{#2}\ifx\temp\empty
                
            \else
                #2 \\ \hline
            \fi
            #3
            \\ \hline
        \end{tabularx}
    }

\begin{document}
    
    \setlength{\parindent}{0cm}

    \begin{tikzpicture} 
        \fill[accent, opacity=1] (0,0) rectangle (21,3);
        \fill[accent, opacity=0.8] (0,-2) rectangle (21,0);
        \fill[accent, opacity=1] (1.5,0.1)
            -- (2,-0.5)
            -- (2.5,0.1)
            -- cycle;
        \node[anchor=east,text=white] (why1) at (18.4,1.5) {
            \huge Linear Regression - "multivariete linear regression" 
        };
        \node[anchor=east,text=white] (why1) at (14,-1) {
            Linear regression with multiple variables is used for classification problems,
        };
        \node[anchor=east,text=white] (why1) at (9.7,-1.4) {
            where the data can be seperated by a straigh line
        };
    \end{tikzpicture}

    % LEFT SIDE OF SHEAT
    \begin{minipage}[t]{.51\textwidth}
        \vspace{1pt}
        \mybox
            {Notes}
            {}
            {
                - uses supervised learning \\ \hline
                Used variables description: \\
                - \(x_{j}^i\) = value of feature j in the \(i^{th}\) traingin example \\
                - \(x^i\) = the column vector of all the feature inputs of the \(i^{th}\) training example \\
                - m = number of traning examples \\
                - n = \(|x^i|\) ; the number of features
            }
        \newline
        \newline
        \newline
        \mybox
            {Hypothesis function}
            {\( h_{\theta}(x) = \Theta_{0} + \Theta_{1}x_{1} + \Theta_{2}x_{2} + \Theta_{3}x_{3} + ... + \Theta_{n}x_{n} \)}
            {
                - equation of a streight line
            }
        \newline
        \newline
        \newline
        \mybox
            {Cost function}
            {\( J(\Theta_{0}, \Theta_{1}) = \frac{1}{2m} \sum\limits_{i=1}^{m} (h\theta(x{(i)}) - y{(i)})^2 \)}
            {
                * meassuring accuracy of hypothesis \\
                * also called "Square error function"
            }
        \newline
    \end{minipage}%
    \begin{minipage}[t]{.51\textwidth}
        \vspace{1pt}
        
    \end{minipage}
    
\end{document}